%%%%%%%%%%%%%%%%%%%%%%%%%%%%%%%%%%%%%%%%%%%%%%%%%%
% TODO
% 11After finishing with the paper and supplementary, name all figures as they are in the paper and in the supplementary materials
% make the captions in normal size

\documentclass[12pt]{article}
% Increase margin width for todos
\setlength{\marginparwidth}{2cm}
\usepackage{comment}
\usepackage{geometry}
\geometry{a4paper, margin=0.6in}
\usepackage{setspace}
\usepackage{amsmath}
\usepackage[most]{tcolorbox}
\usepackage{pifont}
\usepackage{graphicx}
\usepackage{hyperref}
\usepackage{capt-of}
\usepackage{caption}
\usepackage{soul}           % For text highlighting
\usepackage{xcolor}         % For color definitions
\usepackage{todonotes}      % For margin notes
\usepackage{enumitem,amssymb}

% Define custom colors
\definecolor{softtodogreen}{HTML}{90EE90}  % Light green
\setuptodonotes{color=softtodogreen}

\setstretch{1.5}

% Define different colored highlighting commands
\newcommand{\hlgreen}[1]{\sethlcolor{green!30}\hl{#1}}
\newcommand{\hlred}[1]{\sethlcolor{red!30}\hl{#1}}
\newcommand{\hlyellow}[1]{\sethlcolor{yellow!50}\hl{#1}}
\newcommand{\hlblue}[1]{\sethlcolor{blue!20}\hl{#1}}

% Define custom colors
\definecolor{commentcolor}{RGB}{150, 0, 0}
\definecolor{solved_commentcolor}{RGB}{0, 150, 0}
\definecolor{responsecolor}{RGB}{0, 90, 150}


\newcounter{reviewpoint}
\setcounter{reviewpoint}{0}


% Define custom environments with proper alignment
\newenvironment{reviewercomment}
    {\begin{tcolorbox}[width=\linewidth,colback=gray!5,colframe=commentcolor!50,title=Reviewer Comment,left=5pt,right=5pt]}
    {\end{tcolorbox}}

\newenvironment{solved_reviewercomment}
    {\begin{tcolorbox}[width=\linewidth,colback=gray!5,colframe=solved_commentcolor!50,title=Reviewer Comment,left=5pt,right=5pt]}
    {\end{tcolorbox}}
    
\newenvironment{ourresponse}
    {\begin{tcolorbox}[width=\linewidth,breakable,enhanced,colback=gray!5,colframe=responsecolor!50,title=Response,left=5pt,right=5pt]}
    {\end{tcolorbox}}


% For todos, clean definition without duplicates
\newlist{todolist}{itemize}{2}
\setlist[todolist]{label=$\square$}



\begin{document}

\begin{itemize}

\item[$\square$]
Coherent spot size simulation: achive more stastics and try to solve it
\item[$\square$]
Analytic dependence on M (covariance estimator + rank of object) 

\item[$\square$]
Phase retrieval: (a) number of iterations (show plateau) (b) noise + number of realizations (needs good approximation of AC)
\item[$\square$]
Show convergence graph in some measurements + heatmap of iterations as function of M / function of sparsity
\item[$\square$]
Modify [Kang et al., Nat. Commun., 2023] and if possible try and simulate such solution

\item[$\square$]
Mosaicing experiment

\item[$\square$]
SVD and show depndence of singular vector and energy conservation

\item[$\square$]
Figure of different multiplicative factors and their dependence + Explain more the multiplication of different constants 

\item[$\square$]
Heatmap of sparsity and M with noise
\item[\rlap{\raisebox{0.3ex}{\hspace{0.4ex}\scriptsize \ding{51}}}$\square$]
Heatmap of sparsity and M

\end{itemize}
\newpage
\section*{Response to Reviewers}

\subsection*{Reviewer \#1 Comments (from PDF):}

In this work, the authors extend the matrix approach of optical imaging they have been developing with other groups for the last few years to the context of dynamic scattering media. This topic is of great interest because matrix imaging has shown its great potential for overcoming multiple scattering in the long-standing quest towards deep imaging of complex media such as biological tissues. However, it relies by essence on the hypothesis of a static medium. Their approach relies on the fact that, under the isoplanatic hypothesis, dynamic scattering is mathematically analogous to dynamic illumination in static scattering media. From this analogy, they are then able to use the methods developed in previous papers [13] for scattering compensation under dynamic illumination in which the covariance matrix of the reflected wave-field was investigated rather than just the reflection matrix itself. This covariance matrix has already been recently leveraged to extend the scope of matrix imaging to incoherent (fluorescent) imaging by the authors themselves [14]. This paper is a new important demonstration of the versatility and the universality of matrix approaches to wave imaging in complex media. For incoherent imaging, the demonstration of the superiority of matrix imaging with respect to speckle correlation techniques [28,29] is convincing and might have deserved more publicity than just a short section in the Supplementary Material. For coherent imaging applications, the hypothesis of a constant incident wave-field is a bit contradictory with the problem of dynamic scattering and would deserve more discussion.  

As often with this group, this work is technically of excellent quality and its presentation is elegant. The experiments based on rotating diffusers are rather simple and allow the authors to illustrate nicely the concept. However, one can wonder whether the idea proposed by the authors can go beyond imaging through a thin dynamic diffusive layer and to which extent it can be applied to real 3D imaging situations like for biomedical or LiDAR applications. I detail below the main points that, in my opinion, would deserve more discussion before publication.

    \begin{ourresponse}
    We thank the Reviewer for their positive evaluation of our work, recognizing its significance in extending matrix-based imaging to dynamic scattering media and its potential impact.
    \end{ourresponse}

\begin{enumerate}[label=\arabic*.]
    \item \leavevmode\vspace{-\baselineskip}
    \begin{solved_reviewercomment}
        
        For instance, in the coherent imaging experiment depicted in Fig. 5, the approach requires a constant incident wave-field, which is complicated if one wants to image through a dynamic scattering medium. In the present experiment, the authors circumvent this issue by focusing the incident wave-field in the diffuser plane such that the beam waist is contained in one coherence length of the diffuser. However, in practical situations, the scattering medium is three-dimensional, and the condition of a constant incident wave-field becomes impossible to fulfill. In real life, dynamic scattering impacts both the incident and reflected wave-fields. The authors do not discuss really this apparent contradiction between the assumption of a constant incident wave-field and dynamic scattering that will break this invariance.

    \end{solved_reviewercomment}
    \begin{ourresponse}
    We appreciate the reviewer's observation about the apparent contradiction between dynamic scattering and our requirement for constant illumination in the coherent imaging experiment (Fig. 5). 
    
    In our coherent demonstration, we circumvented this challenge by focusing the illumination beam onto a small area of the diffuser (smaller than its correlation length), effectively sampling only a single "phase patch" as the diffuser rotates. However, we recognize this approach is not feasible for volumetric scattering media where maintaining constant illumination becomes challenging.
    
    To address this limitation transparently in our manuscript, we have added the following clarification:
    
    "For coherent imaging through dynamic scattering, our current approach is limited to scenarios where effectively constant illumination can be maintained at the object plane. While demonstrated here with a thin diffuser, volumetric scattering presents additional challenges that require further development."
    
    Beyond this manuscript revision, we propose a hybrid approach that could extend our method to volumetric dynamic media without requiring focused illumination:
    \begin{enumerate}
        \item For each realization of the dynamic scatterer, rapidly acquire multiple ($K \gg 1$) holograms under different speckle illuminations (created with an additional diffuser in the illumination path)
        \item Use coherence gating on each individual hologram to determine the correct object depth and eliminate unwanted reflections
        \item Incoherently sum the intensity patterns to create "macro-frames": $I_m(x,y,z=z_{obj}) = \sum_{k=1}^{K}|E_{m,k}(x,y,z=z_{obj})|^2$ 
        \item Process these macro-frames using I-CLASS as in our incoherent experiments
    \end{enumerate}
    
    This protocol creates effectively uniform illumination through incoherent summing (as in the incoherent experiment with the rapidly rotating diffuser in the illumination path) while preserving the depth-selectivity and reflection-filtering advantages of coherent light. Each resulting macro-frame follows $I_m = P_m * O$, making the problem mathematically equivalent to our incoherent imaging scenario.
    
    Our current experimental demonstration served as a controlled environment to validate the fundamental principles of our approach. The hybrid method described above offers a promising path forward for complex volumetric media while maintaining the benefits of coherent imaging. Additionally, we highlight the applicability of our approach to advanced systems such as flexible multi-core fiber endoscopes (as demonstrated by Choi et al. [Flexible-type ultrathin holographic endoscope for microscopic imaging of unstained biological tissues]), where uniform illumination via single-mode excitation can be straightforwardly achieved.
    \end{ourresponse}






       
    \item \leavevmode\vspace{-\baselineskip}
    \begin{solved_reviewercomment}
        
        Another issue of three-dimensional scattering is that it breaks the isoplanatic assumption on which the current approach relies. The authors mention in the discussion the idea of mosaicking the field-of-view but it somehow requires a roughly focused incident wave-field and only applies to moderate scattering. Another route is a multi-conjugate adaptive optics scheme in which multiple scattering trajectories are rectified by compensating aberrations from a stack of phase screens conjugated with different planes inside the scattering medium [Kang et al., Nat. Commun., 2023]. Would the current approach proposed by the authors be compatible with such a multi-conjugate strategy? It might be relevant to discuss this point in the conclusion.
    \end{solved_reviewercomment}
    \begin{ourresponse}
        We thank the reviewer for highlighting this important limitation. We have demonstrated reconstruction through a thin scattering dynamic layer in our current work. The extension to thick scattering media represents an important challenge and direction for future research.
    
        In conventional CLASS or computational wavefront shaping, thick scatterers can be addressed by using multi-conjugate approaches with multi-layer scattering profiles. However, our approach presents a unique twist: in our formulation, the dynamic medium plays the role that would normally be occupied by the object in conventional CLASS implementations. This role reversal creates both challenges and opportunities.
        
        The modeling and treatment of thick targets proposed in the recent work by Park et al. [https://www.nature.com/articles/s41467-025-56865-z] (Nature Communications, 2024) may offer a pathway forward. In that work, the authors developed techniques for handling thick objects rather than treating them as purely planar. Since in our case the dynamic medium mathematically functions as the "object," adapting these techniques could potentially allow us to extend our approach to volumetric scattering media.
    
        Additionally, another potential approach could be derived from the gradient-descent model-based algorithm recently developed by Haim et al. in their computational holographic wavefront shaping work \cite{haim2024image}. This method's flexibility in computational wavefront correction might offer alternative strategies for handling thick scattering media.
        
        We have added the following paragraph to the discussion section:
        
        "While we have focused our proof-of-principle demonstrations on isoplanatic scattering conditions, extending our approach to thick dynamic scattering media remains an important challenge. Recent advances in modeling thick targets rather than planar ones, such as those presented by Park et al., suggest potential pathways for adaptation to our scenario, where the dynamic medium mathematically plays the role of the object. Future work could explore combining our approach with multi-conjugate modeling, treating thick media as a series of thin scattering layers at different depths, each contributing its own phase and amplitude distortions to the overall scattering process."
    \end{ourresponse}




    \item \leavevmode\vspace{-\baselineskip}
    \begin{solved_reviewercomment}
        One last point which is not really discussed in the paper is the number of required independent frames \(M\) to converge towards a reliable estimation of the object or, in other words, the scaling of the estimator bias with respect to this number \(M\). An analytical formulation of the problem could be performed to derive this number, which is important for future implementations of the method. I guess this number will depend on the object complexity/sparsity.
    \end{solved_reviewercomment}
        

    \begin{ourresponse}

        We appreciate the reviewer's important question about the number of required independent frames for reliable estimation. To address this systematically, we conducted a set of numerical simulations investigating the relationship between reconstruction quality, number of realizations (M), and object sparsity. 

        The theoretical foundation of CTR-CLASS, as established in previous work by Lee et al., indicates a logarithmic dependence of the required number of measurements on the degrees of freedom (i.e., the number of camera pixels $N$). However, this theoretical scaling is modified in practice by additional factors including object complexity (sparsity) and signal-to-noise ratio (SNR). These factors play crucial roles in determining the practical minimum number of frames needed for successful reconstruction.

        We have added a new supplementary section ("Dependence of Reconstruction Quality on Number of Realizations and Object Sparsity") that addresses this question.
        
        % \hlred{Take from I-CLASS / Acusto-optic the paragraph about the theoretical dependence on M which is logarithmic (from the CTR paper) and add that there are additional factors to this theory such as the object complexity and SNR}.
        
        \begin{quote}
            \section{Dependence of Reconstruction Quality on Number of Realizations and Object Sparsity}

            % \hlred{I suggest inserting something about how statistical siginficant is this result? how many experiments are in each heatmap cell? and what is the std between expereimnts (you have 3 random variables, the random object, the shot noise and the PSFs, maybe freeze 2 each time and run it to understand if they have the same effect}.

            
            To study the dependence of the reconstruction quality on the number of realizations and object complexity (sparsity), we numerically simulate the reconstructions at a varying number of random realizations (M) for different object complexities (sparsity) for incoherent and coherent imaging modalities.
            The results of the simulation for varying SNR values at an object sparsity are given in Figure Fig.~\ref{fig:convergence_heatmaps}
            To ensure statistical robustness, each data point in these heatmaps represents the average outcome from 10 independent numerical experiments.

            
            \begin{minipage}{\linewidth}
                \centering
                \includegraphics[width=0.9\textwidth]{figures/figure_S8.pdf}
                \captionof{figure}{\footnotesize \textbf{ Reconstruction fidelity as a function of number of realizations and object sparsity.} 
                Cross-correlation scores between reconstructed and ground truth images are shown as heatmaps for different imaging scenarios. \textbf{a} Example simulated objects with different sparsity levels: $2^{12}$ (left) and $2^8$ (right) bright points. \textbf{b} Incoherent imaging results under different SNR conditions: SNR=5 (left), SNR=10 (middle), and SNR=$\infty$ (right). \textbf{c} Coherent imaging results under the same SNR conditions: SNR=5 (left), SNR=10 (middle), and SNR=$\infty$ (right). The horizontal axis shows the number of realizations (M), and the vertical axis shows object sparsity (represented as the number of bright points). Color scale indicates correlation coefficient from 0 to 1. Each data point represents the average result from 10 distinct numerical experiments with different random objects under identical parameters. For all simulations, a camera pixel count of N=350$^2$ was used.}
            \end{minipage}

            These results demonstrate that sparser objects can be reconstructed with fewer realizations, while complex objects need substantially more measurements. Higher SNR conditions predictably improve reconstruction quality, and coherent imaging appears to require more measurements than incoherent imaging for equivalent fidelity.

                       
        \end{quote}
        
    \end{ourresponse}

\end{enumerate}

\textbf{More Specific Comments:}
\begin{enumerate}[label=\arabic*.]
    \setcounter{enumi}{3}
    \item  \leavevmode\vspace{-\baselineskip}
    \begin{solved_reviewercomment}
        The supplementary Section S2 that shows the impact of correlations between speckle patterns is very interesting and might have deserved a place in the main manuscript. In this Supplementary Section, these correlations arise from energy conservation (phase-distortions). Maybe a singular value (or equivalently eigenvalue) decomposition of the covariance matrix [10] can help in that case? The first singular vector would be less affected by these negative correlations than the CLASS algorithm. Can the authors comment on that? 
    \end{solved_reviewercomment}
    \begin{ourresponse}
        % Add to the main text something like: "Our numerical study shows that correlations that arise from energy conservation in the speckle pattern can be the source for the xxxx. It would be interesting to study whether a singular value decomposition (SVD) of the covariance matrix [10] can help to address these effects?  In this work,  we have heuristically addressed this issue by applying a modulation to the total energy of captured images during post-processing, which effectively mitigates the observed correlation artifacts (see Supplementary Figure S2).

        We thank the reviewer for this insightful suggestion. Following your recommendation, we have added the following discussion to the main manuscript:
        
        \begin{quote}
        \hlred{But citation [10] doesn't perform SVD on the covariance matrix: A. it is not on the covariance but on the full matrix, and B. it is also not on the matrix but on the 'de-scanned' / 'distortion' matrix, which is different. This cannot apply here where the energy conservation creates a problem in the covariance calculation, **before applying algorithm**}
    


        \hlred{Maybe refer to Wonshik's work on SVD (the Science Adavnces one) and show him the effect of pre-processing the measurement matrix with SVD filtering, create a graph where the x axis is the number of non-zero singular values, and the correlation to GT, and see if there is a value smaller than M that will give better correlation}

            Our analysis revealed that energy conservation in phase-only speckle patterns introduces troublesome spatial correlations that can cause background haze in the reconstructions. We addressed this by simply applying varying intensity scaling across frames during post-processing. It would be interesting to study whether a singular value decomposition (SVD) of the covariance matrix [10] can help to address these effects. The full analysis of these correlation effects and our mitigation approach is detailed in Supplementary Section S2 and Supplementary Figure S3.
        \end{quote}
        
        We maintained the detailed analysis in Supplementary Section S2 as you suggested, while bringing the key insight to readers' attention in the main text.

    \end{ourresponse}
    
    \item \leavevmode\vspace{-\baselineskip}
    \begin{solved_reviewercomment}
        The authors show how to circumvent this issue by a multiplication of each frame by a fixed scalar value. This is interesting, but the choice of the multiplication factor from one to two lacks justification. Can the authors provide more details about this choice? 
    \end{solved_reviewercomment}

\begin{ourresponse}
We thank the reviewer for this question about our choice of multiplication factors. To address this, we conducted a systematic study examining how different modulation depths affect reconstruction quality.

We have added a new section to the Supplementary Materials titled "Effect of Modulation Depth on Reconstruction Quality" that demonstrates this trade-off. Here is the full section:

\begin{quote}
\section{Effect of Modulation Depth on Reconstruction Quality}

As discussed in Supplementary Section S2, energy conservation introduces off-diagonal correlations in the covariance matrix that can degrade reconstruction quality. We addressed this by applying variable intensity scaling across the captured frames, with multiplication factors linearly varying from 1 to 2 across frames.

To investigate the optimal range for this modulation, we conducted a study examining how different modulation depths affect reconstruction quality. For each depth value, we applied a linear scaling to our experimental data where the multiplication factor for the $m$-th frame (out of $M$ total frames) was:

\begin{equation}
    f_m = 1 + \frac{m-1}{M-1} \cdot (\alpha -1) = 1, 2, 3, ..., \alpha
\end{equation}

where $\alpha$ controls the total range of modulation with corresponding modulation depth $\frac{\alpha-1}{\alpha+1}$.

\begin{comment}
    We can rewrite this equation directly in terms of the modulation depth $d$ by solving for $\alpha$:

    \begin{align}
    d &= \frac{\alpha-1}{\alpha+1} \\
    d(\alpha+1) &= \alpha-1 \\
    d\alpha + d &= \alpha - 1 \\
    d\alpha - \alpha &= -1 - d \\
    \alpha(d-1) &= -1 - d \\
    \alpha &= \frac{-1-d}{d-1} = \frac{1+d}{1-d}
    \end{align}

    Substituting this into our original equation:
    
    \begin{align}
    f_m &= 1 + \frac{m-1}{M-1} \cdot \left(\frac{1+d}{1-d} - 1\right) \\
    &= 1 + \frac{m-1}{M-1} \cdot \frac{1+d-(1-d)}{1-d} \\
    &= 1 + \frac{m-1}{M-1} \cdot \frac{2d}{1-d}
    \end{align}
    
    Therefore, the multiplication factor for the $m$-th frame in terms of modulation depth $d$ is:
    
    \begin{equation}
        f_m = 1 + \frac{m-1}{M-1} \cdot \frac{2d}{1-d}
    \end{equation}
\end{comment}


\begin{minipage}{\linewidth}
    \centering
    \includegraphics[width=0.85\textwidth]{figures/Figure_S9.pdf}
    \captionof{figure}{\footnotesize \textbf{Effect of modulation depth on reconstruction quality.}
    The similarity score (y-axis) shows the Pearson correlation coefficient between the reconstructed and ground-truth images as modulation depth increases. Example reconstructions demonstrate the progressive elimination of background haze as modulation increases, while at higher modulation depths, some haze begins to appear within the object itself. The red circle indicates our chosen modulation depth, which offers a good trade-off in this balance.}
\end{minipage}
%\hlred{Explain  what is the similarity score in the caption, change y-label to correlation, write in the caption 'Pearson correlation'}

% \hlred{Add $\alpha$ to x label}



Our results show that without adding a modulation (modulation depth = 0), the reconstructions display a significant background haze. As modulation increases, this background haze diminishes progressively. However, at very high modulation depths, while the background continues to reduce, the contrast is somewhat lowered, which may be the result of a imperfect estimation of the MTF in the I-CLASS algorithm.  Modulation depths of 0.3-0.5 used in our main experiments provide an effective trade-off between these competing effects. 
Additional improvements of this point can will be the focus of future work.

% \hlred{What do you mean by 'the reconstruction of I-CLASS does not correct the scattering PSF', of course it corrects, the frames looks much worse (by the way show in the graph an example frame }


% \hlred{Can you clarify on this sentences, what do you mean by independent realizations diminishes}


%\hlred{Maybe show the $PP^T$ for different $\alpha$-s}


\end{quote}
\end{ourresponse}
    
    \item \leavevmode\vspace{-\baselineskip}
    \begin{reviewercomment}
        Even if the I-CLASS and CTR-CLASS have already been presented in previous papers, I would suggest describing them in the Methods section to make the paper self-sufficient. 
    \end{reviewercomment}
    \begin{ourresponse}

        \hlred{\textbf{Elad's comment:}
        
        The descriptions in the methods part are not clear without the matrix description, and the MTF deconvolution part is not relevant here, it should be totally rephrased and organized, and in the supplementary it is also very long and most parts are not relevant to this paper, you don't have to derive all the CLASS algorithm again, I would just write something like $I=P*(O\cdot E_{in}*P_{in})$ and therefore $R=POP$ and CLASS is a matrix unitary decomposition algorithm, after that in few lines about CTR and just bring the final memory efficient term of each phase correction iteration, (maybe write in algorithm pattern as shown in CS papers) And write that the MTF estimation is just like speckle correlation methods. You can take my AO paper as reference, I explained there more briefly}

    
        \hlblue{for now, this is just a copy from your Sci-Adv paper, it has to be eddited}

        \textbf{This is the part copied from, and has to be added to the main manuscript: (again, it is very long, has to be heavily edited.)}
        \begin{quote}
            \subsection*{I-CLASS}
            The full description of the I-CLASS memory-efficient, phase, and amplitude scattering compensation algorithm is fully given in Supplementary Section S4. Here, we provide a concise explanation of the I-CLASS algorithm.
            
            \subsubsection*{Memory-efficient CLASS iterations}
            
            The memory-efficient equivalent method for calculating the CLASS iterations using the $N \times M$ matrix $\hat{\textbf{A}}$ without the explicit computation of the $N \times N$ matrix $\textbf{R}_{virt}$ can be calculated from the following formula for the t-th iteration (see Supplementary S1): 
            
            \begin{eqnarray}
            \vec{z}_{t+1}=\sum_{q=1}^{M}(\tilde{\textbf{A}}_t^* \odot ((\tilde{\textbf{A}}_t^{(ud)^*}*\tilde{\textbf{A}}_t^{(lr)})_{:,M-1} \star \tilde{\textbf{A}}_t^{(ud)}))_{:,q}
            \label{eq:three}
            \end{eqnarray}
            
            \noindent where $\tilde{\textbf{A}}$ as the 2D Fourier transform of $\hat{\textbf{A}}$. Here, $\tilde{\textbf{A}}^{(ud)}$ is the matrix $\tilde{\textbf{A}}$ with each of its columns flipped upside-down, and $\tilde{\textbf{A}}^{(lr)}$ is $\tilde{\textbf{A}}$ with its rows flipped left-to-right. In addition, we denote the element-wise Hadamard product as $\odot$, the 2D convolution as $*$, and the 2D correlation operator as $\star$. $\textbf{X}^{*}$ is the element-wise complex-conjugation operator on the matrix $\textbf{X}$, and $\textbf{X}_{:,q}$ is the q-th column of $\textbf{X}$.
            
            As in the conventional CLASS algorithm \cite{kang17}, throughout the iterative process of the I-CLASS algorithm, we solely update the $\vec{k}$-space (OTF) phase correction with the correction:
            \begin{eqnarray}
            \tilde{\textbf{A}}_{t+1}=diag\{e^{i\frac{\vec{z}_t}{|\vec{z}_t|}}\}\tilde{\textbf{A}}_t
            \end{eqnarray}
            \noindent where the exponential and division operations are element-wise and $t=1...T$ is the iteration number. 
             \subsubsection*{I-CLASS $\vec{k}$-space amplitude correction}
            
            The $\vec{k}$-space amplitude correction in the I-CLASS algorithm consists of two primary steps: (1) estimating the MTF, up to a scaling factor, using $\widehat{MTF} \equiv  \sqrt{diag\{ \tilde{\textbf{R}_{virt}} \}} = \sqrt{\sum_{q=1}^{M} (\tilde{\textbf{A}} \odot \tilde{\textbf{A}}^*)_{:,q}}$ (See supplementary S1). We note that this scaling factor ambiguity inhibits quantitative reconstruction unless the scattering medium transmission is properly estimated. (2) The estimated MTF is then utilized in a regularized Fourier reweighting on each of the captured camera frames after the phase-correction:%last iteration CLASS corrected object $\vec{{O}}_{CLASS}$:
            
            \begin{eqnarray}
            \tilde{\textbf{A}}_{fixed}=\frac{\tilde{\textbf{A}}_T}{\widehat{MTF}+\sigma}
            \end{eqnarray}
            
            %  \begin{eqnarray}
            %      \vec{\tilde{O}}_{I-CLASS_{i}} \equiv \frac{ \vec{\tilde{O}}_{CLASS_{i}}}{\frac{\widehat{MTF}_i}{\max_{q} \widehat{MTF}_q }+\sigma}
            % \end{eqnarray}
            
            \noindent where $\sigma$ is the regularization parameter, and $\tilde{\textbf{A}}_T$ are the phase-corrected frames.  
            
            As a final step, The phase and Fourier-amplitude corrected frames are then reconstructed by inverse Fourier transforming back the corrected Fourier matrix $\tilde{\textbf{A}}_{fixed}$ into $\hat{\textbf{A}}$ and taking the pixel-wise square root of the sum of the columns of $|\hat{\textbf{A}}|^2$ which is equivalent to the square root of the variance image, i.e. the square-root of the scattering-corrected virtual-confocal image (see supplementary S1). This can be written mathematically as $diag\{ \textbf{R}_{virt} \} = \sum_{q=1}^{M} (\hat{\textbf{A}} \odot \hat{\textbf{A}}^*)_{:,q}$, and the I-CLASS corrected object 
            is given by taking an element-wise square root as seen from Eq.~\ref{eq:two}. We note that the I-CLASS final image inherently contains shift ambiguity, where a shift in the PSF can be equally interpreted as a shift in the target object. For convenience, in all the presented reconstructions, the images were co-registered with respect to the widefield reference.

        \end{quote}

        \textbf{This is the part copied from, and has to be added to the supplementary:}
        \begin{quote}
            \section{Modified CLASS Algorithm (I-CLASS)}
            \subsection*{Forward Model}
            
            In the context of incoherent imaging of fluorescence targets through a disordered media, where an object is located within the optical memory effect regime \cite{feng1988correlations}, the image intensity can be described as a convolution of the intensity impulse response (PSF) with the ideal image intensity \cite{goodman2005introduction}.
            For coherent illumination of a fluorescence target object, we obtain the following:
            
            \begin{equation}
            {I}(\vec{r}) = (|{E}_{in}(\vec{r})*{P}^{(E)}_{ill}(\vec{r})|^2{O}(\vec{r}))*{P}(\vec{r})
            \label{eq:1}
            \tag{S1}
            \end{equation}
            
            Where ${P}^{(E)}_{ill}, {P}_{det}$ are the coherent illumination A-PSF and incoherent detection PSF, respectively, and ${O}(\vec{r})$ is the intensity distribution function of the object.
            
            Thus, under conditions of coherent random illuminations, the fluorescence measurement matrix can be constructed in the spatial basis of variable $\vec{r}$, which is sampled and quantized by the $N$ discrete camera pixels. This construction can be expressed mathematically as follows: ${\textbf{A}} = \textbf{P}_{det}\textbf{O}|\textbf{P}^{(E)}_{ill}\textbf{S}^{(E)}|^2$.
            Notably, $\textbf{P}^{(E)}_{ill}$ and $\textbf{P}_{det}$ represent convolution matrices, the optical system input illumination coherent A-PSF and detection incoherent PSF, respectively. Furthermore, the matrix ${\textbf{O}}$ is described by a diagonal matrix, carrying the elements of ${O}(\vec{r})$ on its diagonal, and $\textbf{S}^{(E)}$ contains $M$ columns that represent the $M$ random fields in the illumination plane. For simplicity let us denote the matrix  $\textbf{S} \stackrel{\text{def}} = |\textbf{P}^{(E)}_{ill}\textbf{S}^{(E)}|^2$.
            
            Since $\textbf{S}$ is composed of independent intensity speckle-patterns originating from the same source, one can show that since the patterns are independent, the correlation between a pair of pixels ${a,b}$ and speckle patterns ${i,j}$ can be expressed as $\overline{I_{i_a}I_{j_b}}=\bar{I_a}\bar{I_b}+\delta_{a,b}\delta_{i,j}\bar{I_a}^2$ (using $\overline{I^2}=2\bar{I}^2$) (47).
            
            \noindent This leads to $\overline{(\textbf{S}\textbf{S}^{T})_{i,j}} = \sum^{M-1}_{k=0}\overline{I_{k_i}I_{k_j}} = M\bar{I_i}\bar{I_j}+M\delta_{i,j}\bar{I_i}^2$.
            
            \noindent Thus, considering $\overline{(\textbf{S}\textbf{S}^{T})^2_{i,j}}=\sum^{M-1}_{k,k'=(0,0)}\overline{I_{k_i}I_{k_j}I_{k'_i}I_{k'_j}} = \underbrace{(M^2-M)\overline{I_{i}I_{j}}^2}_{k\neq k'}+M\overline{I^2_{i}I^2_{j}}$ 
            and by using $\overline{I^n_i}=n!\overline{I_i}^n$,  it can be demonstrated that:$\frac{\Delta(\textbf{S}\textbf{S}^{T})_{i,j}}{(\textbf{S}\textbf{S}^{T})_{i,j}} \sim \frac{1}{\sqrt{M}}$
            (where $\Delta X$ signifies the standard deviation of $X$). 
            
            Consequently, we conclude that to a good approximation $\textbf{S}\textbf{S}^{T} \approx Mdiag(\bar{I}\odot \bar{I}) + M\vec{\bar{I}}\vec{\bar{I}}^T$ where $\vec{\bar{I}}$ is the vector that holds the mean intensity values for each pixel ($\odot$ is the element-wise Hadamard product).
            Building on this, we define $\hat{\textbf{S}}\stackrel{\text{def}} = \textbf{S} (\mathbf{I}-\frac{1}{M}\cdot\vec{1}\vec{1}^T)$, where $\vec{1}$ represents a column vector comprising entirely of ones ($\forall i\in[N] :
            \vec{1}_{i}=1$) and $\mathbf{I}$ is the identity matrix. This modification effectively normalizes the speckle data by removing the average intensity contribution from each pixel.
            Importantly, this transformation renders $\hat{\textbf{S}}$ uncorrelated:
            
            \begin{align}
            &\hat{\textbf{S}} \hat{\textbf{S}}^T = \textbf{S}(\mathbf{I}-\frac{1}{M}\cdot\vec{1}\vec{1}^T)^2\textbf{S}^T = \textbf{S}(\mathbf{I}-\frac{1}{M}\cdot\vec{1}\vec{1}^T)\textbf{S}^T \nonumber &\\ &= \textbf{S}\textbf{S}^T-\frac{1}{M}
            \underbrace{(\textbf{S}\vec{1})}_{\approx M\vec{\bar{I}}}
            \underbrace{(\textbf{S}\vec{1})^T}_{\approx M\vec{\bar{I}}^T}\approx Mdiag(\bar{I}\odot \bar{I}) + M\vec{\bar{I}}\vec{\bar{I}}^T- M\vec{\bar{I}}\vec{\bar{I}}^T = Mdiag(\bar{I}\odot \bar{I}) &
            \label{eq:2}
            \tag{S2}
            \end{align}
            leading to the approximation ${\hat{\textbf{S}}} \hat{\textbf{S}}^T \approx \textbf{D}$, for $\textbf{D}\stackrel{\text{def}} =  Mdiag(\bar{I}\odot \bar{I})$.
            
            % We note that this approximation assumes that the spatial autocorrelation of the random speckle pattern is negligible, and can be considered as an additive complex random noise \cite{lee22}, but in fact, as the speckle size is finite, there exists some spatial correlation which smears the object's spectrum \cite{katz14}.
            
            Notably, from a mathematical perspective, it becomes apparent that by subtracting the temporal mean from the measurement matrix $\textbf{A}$, we effectively replicate an illumination pattern with uncorrelated illuminations $\hat{\textbf{S}}$:
            
            \begin{equation}
            \hat{\textbf{A}}\stackrel{\text{def}} = {\textbf{A}}(\mathbf{I}-\frac{1}{M}\cdot\vec{1}\vec{1}^T) = {\textbf{P}_{det}}{\textbf{O}}{\textbf{S}}(\mathbf{I}-\frac{1}{M}\cdot\vec{1}\vec{1}^T)
             = {\textbf{P}_{det}}{\textbf{O}}\hat{\textbf{S}}
             \label{eq:3}
            \tag{S3}
            \end{equation}
            From the above calculation of the covariance matrix, we overall get:
            \begin{equation}
            \textbf{R}_{virt}=\hat{\textbf{A}}\hat{\textbf{A}}^T = {\textbf{P}_{det}}\textbf{O}\hat{\textbf{S}}\hat{\textbf{S}}^T\textbf{O}\textbf{P}_{det}^T \approx \textbf{P}_{det}{\textbf{O}_{eff}} \textbf{P}_{det}^T
            \tag{S4}
            \label{eq:4}
            \end{equation}
            For $\textbf{O}_{eff}\stackrel{\text{def}}=\textbf{O}^2 \textbf{D}$, an incoherent fluorescent analog to a virtual reflection matrix.
            Utilizing the multiplication-convolution relationship in $\vec{k}$-space, we can decompose the Fourier transform of the reflection matrix into: 
            \begin{align}
            \tag{S5}
            \label{eq:5}
            \tilde{\textbf{R}}_{virt}\equiv\mathcal{F}\textbf{R}_{virt}\mathcal{F}^{\dagger} \approx  \mathcal{F}\textbf{P}_{det}{\textbf{O}_{eff}} \textbf{P}_{det}^T\mathcal{F}^{\dagger} = &&  \\ \nonumber 
            \mathcal{F}\textbf{P}_{det}\mathcal{F}^{\dagger}\mathcal{F}\textbf{O}_{eff}\mathcal{F}^{\dagger}(\mathcal{F}\textbf{P}_{det}\mathcal{F}^{\dagger})^{\dagger}={\tilde{\textbf{P}}_{det}}{\tilde{\textbf{O}}_{eff}}{\tilde{\textbf{P}}^{\dagger}_{det}}
            \end{align}
            (where $\mathcal{F}$ represents the unitary DFT matrix).
            
            
            In this Fourier domain representation, $\tilde{\textbf{P}}_{det}$ which is the Fourier transform of $\textbf{P}_{det}$, takes on a diagonal form, reflecting the fact that the convolution operations in real space become simple multiplications in Fourier space. In the case of a unitary phase-only distortion (which is the model in the coherent case), these diagonal matrices contain the random phases $e^{i\phi_{det}(\vec{k})}$ and $e^{-i\phi_{det}(\vec{k})}$, respectively, which the CLASS algorithm aims to correct. On the other hand, the matrix $\tilde{\textbf{O}}_{eff}$ which is the Fourier transform of $\textbf{O}_{eff}$, now adopts the form of a Toeplitz convolution matrix, with the Fourier components of the object.
            
            \subsection*{CLASS algorithm}
            
            The CLASS algorithm \cite{kang17} aims to correct the distortion, $\tilde{\textbf{P}}_{det}$, and retrieve the ideal image intensity ${\tilde{\textbf{O}}_{eff}}$. Mathematically, this requires finding the inverse of $\tilde{\textbf{P}}_{det}$ which, being modeled as a unitary transformation with only phase aberrations, is equivalent to finding $\tilde{\textbf{P}}_{det}^{\dagger}$.
            We note that very recently, a tutorial with detailed algorithm implementations of CLASS was published \cite{kang2024implementation}. 
            %The algorithm aims to minimize the difference between the k-space $\tilde{\textbf{R}}$ and $\tilde{\textbf{P}}_{det}^{\dagger}\tilde{\textbf{P}}_{det}\tilde{\textbf{O}}_{eff}\tilde{\textbf{P}}_{det}^{\dagger}\tilde{\textbf{P}}_{det} = \tilde{\textbf{O}}_{eff}$, which is a Toeplitz matrix. 
            % The second perspective involves maximizing the diagonal elements of $\textbf{R}$ in $r$-space, which represent the "effective confocal energy". Both perspectives are entirely equivalent, as evidenced by their Fourier relations.
            %The problem is addressed in k-space in the CLASS algorithm, where $\tilde{\textbf{P}}_{det}$ is modeled as a diagonal matrix containing only phase components on its diagonal.
            
            The standard CLASS algorithm, handling a full system reflection matrix in the form of  $\textbf{R}={\tilde{\textbf{P}}_{det}}{\tilde{\textbf{O}}}{\tilde{\textbf{P}}_{ill}}$, addresses a single distortion during each iteration, alternately handling $\tilde{\textbf{P}}_{det}$ and $\tilde{\textbf{P}}_{ill}$. However, in CTR-CLASS \cite{lee22} where $\tilde{\textbf{P}}_{ill} = \tilde{\textbf{P}}_{det}$, exclusively tackles the "right" matrix $\tilde{\textbf{P}}_{det}$ in each iteration and fixes the matrix from both sides.
            
            In each iteration of the CLASS algorithm, only a single aberration is corrected—either the output (detection) or the input (illumination) aberration. Specifically, when correcting the input aberration, the output aberration is temporarily neglected, and vice versa. The convergence of CLASS assumes correlation in the uncorrected aberration in each iteration \cite{kang17}. Therefore if we displace each column proportionally to its index, the resultant matrix should approximately exhibit columns of equal values which contain the spectrum of $\tilde{O}_{eff}(\vec{k})$, albeit with distinct global phases for each column. Consequently, if we compute the average of these columns, we obtain a reasonably accurate estimation of $\tilde{O}_{eff}(\vec{k})$, given that the phase averages out.\\
            Now that we possess an approximation of $\tilde{O}_{eff}(\vec{k})$, we can determine the overall phase for each column. This can be achieved by calculating the correlation angle between the j-th column and the estimated $\hat{\tilde{O}}_{eff}(\vec{k})$:
            
            \begin{equation}
            \hat{\phi}_{det}({k_{j})} = arg\{<\hat{\tilde{O}}_{eff}^*(\vec{k})\tilde{O}_{eff}(\vec{k})e^{i\phi_{det}(k_{j})}>_k\}
            \tag{S6}
            \label{eq:6}
            \end{equation}
            By incorporating the terms $e^{-i\hat{\phi}_{det}(\vec{k})}$ into a diagonal matrix $\hat{\tilde{\textbf{P}}}_{det}$ we can fix $\tilde{\textbf{R}}$ by $\tilde{\textbf{R}}_t=\hat{\tilde{\textbf{P}}}_{det}^{\dagger}\tilde{\textbf{R}}_{t-1}\hat{\tilde{\textbf{P}}}_{det}$. This iterative approach consistently leads to the convergence of the correct phase correction.
            We now write the iteration in a matrix notation by defining the following:
            \begin{equation}
            \vec{z}\stackrel{\text{def}} =  \tilde{\textbf{R}}^{T}_{n_s}(\tilde{\textbf{R}}_{n_s}\vec{1})^*=(\tilde{\textbf{R}}^{\dagger}_{n_s}\tilde{\textbf{R}}_{n_s}\vec{1})^*=\vec{1}^T\tilde{\textbf{R}}^{\dagger}_{n_s}\tilde{\textbf{R}}_{n_s}
            \tag{S7}
            \label{eq:7}
            \end{equation}
            
            With this notation, we obtain:  $\hat{\phi}_{det}({k_{j})}=\frac{\vec{z}_j}{|\vec{z}_j|}$
            
            Here, $\tilde{\textbf{R}}_{n_s}$ represents the shifted version of $\tilde{\textbf{R}}^{(ud)}$, which can be expressed as:
            \begin{equation}
            \forall i\in[2N-1] \forall j\in[N]:
            \tilde{\textbf{R}}_{n_{s_{i,j}}}=\sum^{N-1}_{a=0}\sum^{N-1}_{b=0}S_{i,j,a,b}\tilde{\textbf{R}}^{(ud)}_{n_{a,b}}
            \tag{S8}
            \label{eq:8}
            \end{equation}
            Using $S_{i,j,a,b} = \delta_{j,b}\delta_{i,a+b}$, where $N$ represents the camera pixel count. Additionally, we use $\tilde{\textbf{R}}^{(ud)}_{n_{i,j}} = \tilde{\textbf{R}}_{N-1-i,j}$ i.e. flipping each column of $\tilde{\textbf{R}}$, a measure that was taken for the sake of index convenience.
            (By denoting $\textbf{R}^{(ud)}_{{i,j}} = \textbf{R}_{N-1-i,j}$ we can get $\tilde{\textbf{R}}^{(ud)} = \tilde{\textbf{A}}^{(ud)}\tilde{\textbf{A}}^\dagger$)
            % 
            
            
            
            \subsection*{Memory Complexity}
            Given that we capture only $M$ images from distinct random illuminations, each containing $N$ pixels, where $M<<N$, it is advantageous to avoid explicitly constructing the matrix $\textbf{R}_{virt}$ as ${\hat{\textbf{A}}}{\hat{\textbf{A}}}^{\dagger}$, an $N$x$N$, matrix, due to its high memory demand. All the necessary information should be encapsulated within $\hat{\textbf{A}}$, which is of size $N$x$M$. As a result, we show an equivalent mathematical expression for a CLASS iteration using $\hat{\textbf{A}}$ without directly computing $\textbf{R}_{virt}$.
            
            We note that since the DFT matrix \( \mathcal{F} \) is unitary, we can apply a two-dimensional Fourier transform to the measured images before including them in \( {\hat{\textbf{A}}} \), thus obtaining \( \tilde{\textbf{R}}_{virt} = \mathcal{F}\textbf{R}_{virt}\mathcal{F}^{\dagger} = \mathcal{F}{\hat{\textbf{A}}}{\hat{\textbf{A}}}^{\dagger}\mathcal{F}^{\dagger} = \mathcal{F}{\hat{\textbf{A}}}\mathcal{F}^T(\mathcal{F}{\hat{\textbf{A}}}\mathcal{F}^T)^{\dagger} \). Therefore, we can concentrate on $\tilde{\textbf{A}}=\mathcal{F}{\hat{\textbf{A}}}\mathcal{F}^{T}$.
            
            First, we find the vector elements of $\vec{z} = \vec{1}^T\tilde{\textbf{R}}^{\dagger}_{n_s}\tilde{\textbf{R}}_{n_s}$ (Eq.~\ref{eq:7}):
            
            \begin{align}
            & z_j = \sum^{N-1}_{i=0}(\tilde{\textbf{R}}^{\dagger}_{s}\tilde{\textbf{R}}_{s})_{i,j} = \sum^{N-1}_{i=0} \sum^{2N-2}_{m=0}(\tilde{\textbf{R}}^{*}_{s_{m,i}}\tilde{\textbf{R}}_{s_{m,j}}) \nonumber &\\&\underset{(S8)}{=}  \sum^{N-1}_{i=0}\sum^{2N-2}_{m=0}\sum^{N-1}_{k=0}\sum^{N-1}_{l=0}\sum^{N-1}_{a=0}\sum^{N-1}_{b=0}S_{m,j,a,b}\tilde{\textbf{R}}^{(ud)}_{a,b}S_{m,i,k,l}\tilde{\textbf{R}}^{(ud)^*}_{k,l} \nonumber &\\ 
            &= \sum^{N-1}_{i=0}\sum^{N-1}_{k=0}\sum^{N-1}_{l=0}\sum^{N-1}_{a=0}\sum^{N-1}_{b=0}\tilde{\textbf{R}}^{(ud)^*}_{k,l}\tilde{\textbf{R}}^{(ud)}_{a,b}\sum^{2N-2}_{m=0}S_{m,j,a,b}S_{m,i,k,l} \nonumber &\\ 
            &= \sum^{N-1}_{i=0}\sum^{N-1}_{k=0}\sum^{N-1}_{l=0}\sum^{N-1}_{a=0}\sum^{N-1}_{b=0}\tilde{\textbf{R}}^{(ud)^*}_{k,l}\tilde{\textbf{R}}^{(ud)}_{a,b}\delta_{j,b}\delta_{i,l}\sum^{2N-2}_{m=0}\delta_{m,a+b}\delta_{m,k+l} \nonumber &\\ 
            &= (2N-1)\sum^{N-1}_{i=0}\sum^{N-1}_{k=0}\sum^{N-1}_{a=0}\tilde{\textbf{R}}^{(ud)}_{a,j}\tilde{\textbf{R}}^{(ud)^*}_{k,i}\delta_{a+j,k+i} \nonumber &\\ 
            &\propto \sum^{N-1}_{i=0}\sum^{N-1}_{k=0}\sum^{N-1}_{a=0}\sum^{M-1}_{n=0}\sum^{M-1}_{m=0}\tilde{\textbf{A}}^{(ud)}_{a,n}\tilde{\textbf{A}}^*_{j,n}\tilde{\textbf{A}}^{(ud)^*}_{k,m}\tilde{\textbf{A}}_{i,m}\delta_{a+j,k+i} \nonumber &\\ 
            &= \sum^{N-1}_{k=0}\sum^{N-1}_{a=0}\sum^{M-1}_{n=0}\tilde{\textbf{A}}^{(ud)}_{a,n}\tilde{\textbf{A}}^*_{j,n}\sum^{M-1}_{m=0}\tilde{\textbf{A}}^{(ud)^*}_{k,m}\tilde{\textbf{A}}_{a+j-k,m}\sum^{N-1}_{i=0}\delta_{a+j,k+i}
             \nonumber &\\ 
            &= \sum^{N-1}_{a=0}\langle \vec{\tilde{\textbf{A}}}_{j,:},\vec{\tilde{\textbf{A}}}^{(ud)}_{a,:} \rangle\sum^{min(N-1,a+j)}_{k=max(0,a+j-(N-1))}\langle \vec{\tilde{\textbf{A}}}^{(ud)}_{k,:},\vec{\tilde{\textbf{A}}}_{a+j-k,:} \rangle &
            \tag{S9}
            \label{eq:9}
            \end{align}
            
            where we denote $\vec{\tilde{\textbf{A}}}_{c,:}$ as the c-th column of $\tilde{\textbf{A}}$. By denoting ${\textbf{A}}^{(lr)}_{a,b}={\textbf{A}}_{a,M-1-b}$ and using full-sized convolution (with zero-padding) yields:
            \begin{equation}
            ({\textbf{A}}^{(ud)^*}*{\textbf{A}}^{(lr)})_{a+j,M-1} = \sum^{min(N-1,a+j)}_{k=max(0,a+j-(N-1))}\langle {\textbf{A}}^{(ud)}_{k,:},{\textbf{A}}_{a+j-k,:} \rangle \label{eq:10} \tag{S10}
            \end{equation}
            By plugging Eq.~\ref{eq:10} into Eq.~\ref{eq:9}, we obtain:
            
            \begin{equation}
            \vec{z}=\sum^{N-1}_{a=0}(\tilde{\textbf{A}}^*\vec{\tilde{\textbf{A}}}_{a,:})\odot(\tilde{\textbf{A}}^{(ud)^*}*\tilde{\textbf{A}}^{(lr)})_{a:a+N,M-1}
            \label{eq:11} \tag{S11}
            \end{equation}
            Where $\odot$ denotes the Hadamard Product (element-wise multiplication) between two vectors. And the phase correction for each iteration is received by $\hat{\vec{\phi}} = \frac{\vec{z}}{|\vec{z}|}$ (element-wise division).
            
            To furtherly remove the sum over $a$, we denote $\vec{b}\stackrel{\text{def}} = (\tilde{\textbf{A}}^{(ud)^*}*\tilde{\textbf{A}}^{(lr)})_{:,M-1}$, rewriting the expression of $\vec{z}$ as:
            
            \begin{equation}
            z_i=(\sum^{N-1}_{a=0}(\tilde{\textbf{A}}^*\vec{\tilde{\textbf{A}}}_{a,:})\odot\vec{b}_{a:a+N})_i =\sum^{M-1}_{j=0} \tilde{\textbf{A}}^*_{i,j}\sum^{N-1}_{a=0}\tilde{\textbf{A}}^{(ud)}_{a,j}\vec{b}_{a+i}=\sum^{M-1}_{j=0} \tilde{\textbf{A}}^*_{i,j}(\vec{b} \star \tilde{\textbf{A}}^{(ud)})_{i,j}
            \label{eq:12}  \tag{S12}
            \end{equation}
            where $\star$ is a 2D cross-correlation.
            Finally, the correction can be written simply as:
            \begin{equation}
            \vec{z}_{t+1}=(\tilde{\textbf{A}}_t^* \odot ((\tilde{\textbf{A}}_t^{(ud)^*}*\tilde{\textbf{A}}_t^{(lr)})_{:,M-1} \star \tilde{\textbf{A}}_t^{(ud)}))\vec{1}
            \label{eq:13}  \tag{S13}
            \end{equation}
            Therefore, after each iteration $t = 1...T$, we update the matrix using the formula
            $\tilde{\textbf{A}}_{t+1}=diag\{\frac{\vec{z}_{t+1}}{|\vec{z}_{t+1}|}\}\tilde{\textbf{A}}_t$, where both the exponential function and the division are applied element-wise  (the function $diag\{\vec{x}\}$ constructs a diagonal matrix with the elements of vector $\vec{x}$ placed along its diagonal).
            Since building this correction diagonal matrix is memory inefficient, we correct each q-th column of $\tilde{\textbf{A}}_t$ with the element-wise multiplication: $ \frac{\vec{z}_{t+1}}{|\vec{z}_{t+1}|} \odot \tilde{\textbf{A}}_{t_{:,q}}$.
            
            After all iterations, to restore the object approximation $\hat{O}_{eff}=diag\{\textbf{R}_{virt}\}$ we then calculate:
            \begin{align}
            \label{eq:14}  \tag{S14}
            &\hat{O}_{eff_i}=\textbf{R}_{virt_{i,i}}=\mathcal{F}^{\dagger}\tilde{\textbf{R}}_{virt}\mathcal{F}_{i,i}=\mathcal{F}^{\dagger}\tilde{\textbf{A}}\tilde{\textbf{A}}^\dagger\mathcal{F}_{i,i} &\\ 
            &= \mathcal{F}^{\dagger}\tilde{\textbf{A}}\mathcal{F}\mathcal{F}^{\dagger}\tilde{\textbf{A}}^\dagger\mathcal{F}_{i,i} \nonumber = \sum^{N-1}_{m=0}|\mathcal{F}^{\dagger}\tilde{\textbf{A}}\mathcal{F}|^2_{i,m} &
            \end{align}
            Overall, the object can be written as: $\vec{{O}}_{eff} =|iFFT2(\tilde{\textbf{A}})|^2\vec{1}$, which is the sum over the columns of $|{\hat{\textbf{A}}}|^2$.
            We conclude that the improvement in the memory-efficient I-CLASS algorithm now has memory complexity $O(MN)$ instead of $O(N^2)$, an improvement of $N/M$, which in our experiments is $\sim$ $10^4$.
            
            
            \subsection*{Regularized Fourier Reweighting}
            
            In the incoherent imaging scenario mentioned in the main text, non-unitary distortion and amplitude modulation cause 'haze' in the phase-corrected image. Thus, in estimating the Modulation Transfer Function (MTF), the absolute value of the Optical Transfer Function (OTF), is essential. 
            
            In the preceding discussion, and as demonstrated in Eq.~\ref{eq:5}, we observe that ${\tilde{\textbf{P}}_{det}}$ is structured as a diagonal matrix containing on its diagonal  ${\tilde{\textbf{P}}_{det_{i,i}}}=MTF(k_i)e^{i\phi_{det}(k_i)}$, while ${\tilde{\textbf{O}}_{eff}}$ takes the form of a convolution matrix. This allows us to express the elements of $\tilde{\textbf{R}}$ as follows:
            
            \begin{equation}
            \tilde{\textbf{R}}_{i,j} = \sum_{a,b}{\tilde{\textbf{P}}_{det_{i,a}}} \tilde{\textbf{O}}_{eff_{a,b}}{\tilde{\textbf{P}}^{\dagger}_{det_{b,j}}}=\sum_{a,b}{\tilde{\textbf{P}}_{det_{i,i}}} \delta_{i,a} \vec{\tilde{O}}_{eff_{a-b}}{\tilde{\textbf{P}}^*_{det_{j,j}}} \delta_{b,j}
            \label{eq:15}  \tag{S15}
            \end{equation}
            This formulation leads to the following expression for the diagonal elements:
            
            \begin{equation}
            \tilde{\textbf{R}}_{i,i} = \vec{\tilde{O}}(0) {\tilde{\textbf{P}}_{det_{i,i}}}{\tilde{\textbf{P}}^*_{det_{i,i}}} \propto |{\tilde{\textbf{P}}_{det_{i,i}}}|^2
            \label{eq:16}  \tag{S16}
            \end{equation}
            Consequently, this relationship enables the estimation of the MTF up to a scaling factor, by:
            $\widehat{MTF} \equiv \sqrt{diag\{\tilde{\textbf{R}}\}}$. %We note that this scaling factor limits the quantitative reconstruction. 
            Similarly to the approach outlined in Eq.~\ref{eq:14}, the MTF can be directly computed from $\tilde{\textbf{A}}$, bypassing the need for explicit calculation of $\tilde{\textbf{R}}$. This is achieved by summing over the columns of $|{\tilde{\textbf{A}}}|^2$.
            
            This can also be derived in an integral form, which is perhaps more natural for describing the forward optical physical model. Given the forward model \ref{eq:1}  $I_{m}(\vec{r}) = P(\vec{r'}) * (O(\vec{r'}) \cdot S_{m}(\vec{r'}))|_{\vec{r}}$ where $O(\vec{r})$ is the fluorophore distribution (the object function) and $S_{m}(\vec{r})$ is the illumination speckle pattern on the object plane. When assuming, as mentioned above, a delta-like speckle covariance: $C_S(\vec{r}_1,\vec{r}_2) \stackrel{\text{def}} = \langle S_m(\vec{r}_1)S_m(\vec{r}_2) \rangle_m - \langle S_m(\vec{r}_1) \rangle_m \langle S_m(\vec{r}_2) \rangle_m \propto\delta(\vec{r}_1-\vec{r}_2)$, and defining $\hat{S}_m(\vec{r})=S_m(\vec{r})-\langle S_{m'}(\vec{r})\rangle_{m'}$, we obtain $\langle \hat{S}_m(\vec{r}_1)\hat{S}_m(\vec{r}_2) \rangle_m =C_S(\vec{r}_1,\vec{r}_2) \propto\delta(\vec{r}_1-\vec{r}_2)$. Additionally, by defining $\tilde{\hat{I}}_m(\vec{r})\stackrel{\text{def}} = I_m(\vec{r})-\langle I_{m'}(\vec{r})\rangle_{m'}=P(\vec{r'}) * (O(\vec{r'}) \cdot (S_{m}(\vec{r'})-\langle S_{m'}(\vec{r})\rangle_{m'}))=P (\vec{r'})* (O(\vec{r'}) \cdot \hat{S}_{m}(\vec{r'}))|_{\vec{r}}$, we can apply the multiplication-convolution property of the Fourier transform to get: $\tilde{\hat{I}}_m(\vec{k})\stackrel{\text{def}} = OTF(\vec{k})\cdot(\tilde{O}(\vec{k})*\tilde{\hat{S}}_m(\vec{k}))$. Therefore, the spectrum covariance can be expressed as:
            \begin{flalign}
            &C_I(\vec{k}_1,\vec{k}_2) \stackrel{\text{def}} =  \langle \tilde{\hat{I}}_m(\vec{k}_1)\tilde{\hat{I}}^*_m(\vec{k}_2)   \rangle_m  \nonumber \\& =  
            \langle (OTF(\vec{k}_1)\cdot(\tilde{O}(\vec{k}_1)*\tilde{\hat{S}}_m(\vec{k}_1))(OTF(\vec{k}_2)\cdot(\tilde{O}(\vec{k}_2)*\tilde{\hat{S}}_m(\vec{k}_2)))^* \rangle_m \nonumber  \\& =\langle OTF(\vec{k}_1)OTF^*(\vec{k}_2)(\tilde{O}(\vec{k}_1)*\tilde{\hat{S}}_m(\vec{k}_1))\cdot(\tilde{O}^*(\vec{k}_2)*\tilde{\hat{S}}^*_m(\vec{k}_2)) \rangle_m \nonumber  \\& =\langle OTF(\vec{k}_1)OTF^*(\vec{k}_2)\iint\tilde{O} (\vec{k}'-\vec{k}_1)\tilde{\hat{S}}_m(\vec{k}')d\vec{k}'\iint\tilde{O}^* (\vec{k}''-\vec{k}_2)\tilde{\hat{S}}^*_m(\vec{k}'')d\vec{k}'' \rangle_m  \nonumber  \\& =OTF(\vec{k}_1)OTF^*(\vec{k}_2)\iint\tilde{O} (\vec{k}'-\vec{k}_1)d\vec{k}'\iint\tilde{O}^* (\vec{k}''-\vec{k}_2) \nonumber  \\& \iiiint e^{-i\vec{k}'\cdot \vec{r}'}e^{i\vec{k}''\cdot \vec{r}''}\underbrace{\langle{\hat{S}_m(\vec{r}')}{\hat{S}_m(\vec{r}'')}}_{\delta(\vec{r}'-\vec{r}'' )}\rangle_md\vec{k}''d\vec{r}'d\vec{r}'' \nonumber  \\&= OTF(\vec{k}_1)OTF^*(\vec{k}_2)\iint\tilde{O} (\vec{k}'-\vec{k}_1)d\vec{k}'\iint\tilde{O}^* (\vec{k}''-\vec{k}_2) 
            d\vec{k}''\underbrace{\iint e^{-i(\vec{k}'-\vec{k}'')\cdot \vec{r}'}d\vec{r}'}_{\delta(\vec{k}'-\vec{k}'')} \nonumber  \\&= OTF(\vec{k}_1)OTF^*(\vec{k}_2)\iint\tilde{O} (\vec{k}'-\vec{k}_1)\tilde{O}^* (\vec{k}'-\vec{k}_2)d\vec{k}'\nonumber  \\&= OTF(\vec{k}_1)OTF^*(\vec{k}_2)\iint\tilde{O} (\vec{k}'')\tilde{O}^* (\vec{k}''+\vec{k}_1-\vec{k}_2)d\vec{k}'' 
            \label{eq:17}  \tag{S17}
            \end{flalign}
            Thus, similarly to Eq. \ref{eq:16}, we obtain on the 'diagonal' of the covariance term:
            $C_I(\vec{k},\vec{k}) = |OTF(\vec{k})|^2\iint
            |\tilde{O} (\vec{k}'')
            |^2d\vec{k}''\propto |OTF(\vec{k})|^2$, and estimate the MTF by $\widehat{MTF}(\vec{k})\equiv\sqrt{C_I(\vec{k},\vec{k})}$
            
            
            Overall, the $\vec{k}$-space amplitude correction is estimated by taking the amplitude correction $\widehat{MTF}$ as input for a regularized Fourier reweighting process, where each camera frame is reweighted independently with the same amplitude correction. %Fourier component of $\vec{{O}}_{CLASS}$, denoted as  $\vec{\tilde{O}}_{CLASS}$, is divided by a factor given by:
            %\begin{equation}
            %\vec{\tilde{O}}_{I-CLASS_{i}} \equiv \frac{ \vec{\tilde{O}}_{CLASS_{i}}}{\frac{\widehat{MTF}_i}{\max_{q} \widehat{MTF}_q }+\sigma}
            %\label{eq:17}   \tag{S17}
            %\end{equation}
            \begin{equation}
            \tilde{\textbf{A}}_{fixed}=\frac{\tilde{\textbf{A}}_T}{\widehat{MTF}+\sigma}
            \label{eq:18}   \tag{S18}
            \end{equation}
            
            \noindent where $\sigma$ is the regularization parameter, and $\tilde{\textbf{A}}_T$ are the phase-corrected frames.
            Although one can utilize various deconvolution methods, such as Wiener or Richardson-Lucy, our empirical observations have shown that regularized deconvolution yields superior results using the regularization of Eq.\ref{eq:18}.
            
            The impact of different regularization parameters is demonstrated through I-CLASS corrections in Fig.~\ref{fig1_supp}. Fig.~\ref{fig1_supp}A features the CLASS correction, similar to taking the regularization parameter $\sigma $ to $\infty$. Fig.~\ref{fig1_supp}B displays the result of an excessively high regularization parameter, leading to haze in the image. Fig.~\ref{fig1_supp}C demonstrates an optimal balance between noise reduction and the preservation of high-frequency details. Lastly, Fig.~\ref{fig1_supp}D presents the effect of setting the regularization parameter too low, resulting in the inclusion of noisy high-frequency bands.

            \begin{minipage}{\linewidth}
            	\centering
            	\includegraphics [width=0.95\textwidth]{figures/figure_S5.pdf}
                \captionof{figure}{\footnotesize\textbf{Regularized Fourier-reweighting with varying regularization parameter}. (A) Corrected confocal image via phase-only OTF correction shows substantial hazing due to uncorrected amplitude attenuation. (B) The result of a too-high regularization parameter resulted in the inclusion of haze. (C) Shows the scenario where the regularization parameter optimally reduces noise while preserving high-frequency details. (D) Illustrates the effect when the regularization parameter (sigma) is set too low, including noisy high-frequency bands. Scale bars, 200 $\mu m$. Insets in (A-D) compare the marked square areas.}\label{fig1_supp}
            \end{minipage} 
            \noindent In summary, the I-CLASS algorithm proceeds as follows: First, it applies the memory-efficient version of the CTR-CLASS phase correction to all camera frames. Next, it estimates the MTF from the diagonal of the Fourier-transformed reflection matrix. Then, it performs Fourier-reweighting on all camera frames using the estimated MTF. Lastly, it calculates the square root of the variance of the corrected frames, which is the final I-CLASS corrected image.
            % In summary, the I-CLASS algorithm proceeds as follows in three steps: initially, it employs the memory-efficient version of the CTR-CLASS phase correction on all camera frames, followed by estimating the MTF from the diagonal of the Fourier-transformed reflection matrix applying deconvolution with the estimated MTF on all camera frames and finally calculating the square root of the variance image.


        \end{quote}
        
    \end{ourresponse}
    
    
    \item \leavevmode\vspace{-\baselineskip}
    \begin{solved_reviewercomment}
        The same remark holds for the Fresnel operators applied at different stages of the post-processing for the holographic experiment.
    \end{solved_reviewercomment}

    % \hlred{Why Fresnel and not ASM?}
    

    \begin{ourresponse}
        We thank the reviewer for this helpful suggestion. The revised manuscript now includes a detailed description of the Fresnel propagation operator used in the holographic reconstruction. This has been added to the \textbf{Methods} section titled "Fresnel Propagation via Fourier-Domain Transfer Function".
        \begin{quote}
            \subsection*{Fresnel Propagation via Fourier-Domain Transfer Function}
            \label{Fresnel_propagation}
            In Fig.~\ref{fig5}, we present the object field located at the physical object plane, $z_{\text{obj}}$. However,since we measure the field on the diffuser plane, the I-CLASS algorithm reconstructs the complex field in the plane of the scattering layer, denoted as $E_o(x, y, z_{\text{scatt}})$. To visualize the field at the object plane, we propagate the reconstructed field from the diffuser plane to the object plane using Fresnel propagation under the paraxial approximation.
            
            This propagation is efficiently implemented in the Fourier domain using the Fresnel transfer function:
            
            \begin{equation}
                E_o(x, y, z_{\text{obj}}) = \mathcal{F}^{-1} \left\{ \tilde{E}_o(f_x, f_y, z_{\text{scatt}}) \cdot H(f_x, f_y; \Delta z) \right\}
            \end{equation}
            
            where:
            - $ \tilde{E}_o(f_x, f_y, z_{\text{scatt}}) = \mathcal{F}\{ E_o(x, y, z_{\text{scatt}}) \} $ is the 2D Fourier transform of the reconstructed field,
            - $ H(f_x, f_y; \Delta z) $ is the Fresnel transfer function,
            - $ \Delta z = z_{\text{obj}} - z_{\text{scatt}} $ is the propagation distance,
            - $ \mathcal{F} $ and $ \mathcal{F}^{-1} $ denote the 2D Fourier and inverse Fourier transforms.
            
            The Fresnel transfer function in terms of spatial frequency is:
            
            \begin{equation}
                H(f_x, f_y; \Delta z) = \exp\left[ i \frac{2\pi \Delta z}{\lambda} \right] \cdot \exp\left[ -i \pi \lambda \Delta z (f_x^2 + f_y^2) \right]
            \end{equation}
            
            Here:
            - $ \lambda $ is the illumination wavelength,
            - $ (f_x, f_y) $ are the spatial frequency coordinates corresponding to the real-space axes $ (x, y) $.
            
            This formulation supports forward and backward propagation by simply changing the sign of $ \Delta z $, and is especially suitable for numerical implementation via Fast Fourier Transforms.
            
        \end{quote}

                
    \end{ourresponse}
    

    
    \item \leavevmode\vspace{-\baselineskip}
    \begin{solved_reviewercomment}
        In that respect, it is not really clear to me why, in the holographic experiment, the I-CLASS reconstructed object field is obtained in the scattering layer plane? A more detailed description of the method is needed.
    \end{solved_reviewercomment}
        
    
    \begin{ourresponse}
        \begin{comment}
        \hlred{I would add that we measure on the diffuser plane and hence the multiplicative factor is with the propagated object, so that is the object that is reconstructed}

    
        We thank the reviewer for raising this important point.
        
        In conventional CLASS/I-CLASS implementations, the output of the algorithm is the scattering layer function at the pupil of the optical system, that is, at the Fourier plane of the target. CLASS algorithms specifically look for correlations in the k-space, where the reflection matrix becomes diagonal
        \hlred{the reflection matrix is definitely not diagonal...} under the isoplanatic approximation and outputs the phase mask in this space.
        
        In our work, the roles are reversed: the object plays the role of the scattering medium. Consequently, the I-CLASS "scattering function" output is effectively the object function in k-space. This is why our reconstruction naturally occurs in the plane corresponding to the scattering layer.


        \hlred{Write equations... original CLASS $POP\quad \overrightarrow{}\tilde{P}\tilde{O}\tilde{P}$, Here $O_{prop}PO_{prop}$ where $O_{prop}$ is diagonal and $P$ is the envelope, no Fourier needed}
        
        This approach creates no fundamental limitation for our imaging capability. Since we recover the complete complex wavefront, we can numerically propagate it to any desired axial plane using Fresnel propagation, as detailed in the Methods section.
        \end{comment}
        We thank the reviewer for raising this important point.
    
        Reconstructing in the diffuser plane is a matter of convenience rather than a fundamental requirement. In principle, we could have followed the same route as in the incoherent experiments—image the object plane, apply a 2-D Fourier transform, run CLASS, and obtain the restored $\tilde{O}(\vec{k})$. Our choice to work in the diffuser plane for the coherent case was based on practical considerations related to optical efficiency.
        
        In conventional CLASS/I-CLASS implementations, the output of the algorithm is the scattering layer function at the pupil of the optical system, that is, at the Fourier plane of the target. CLASS algorithms specifically look for correlations in k-space, where the transmission matrix of the scattering layer becomes diagonally structured under the isoplanatic approximation, and outputs the phase mask in this space.
        
        Within a single isoplanatic patch, the measured field obeys $I_m(\vec{r})=P_m(\vec{r})*O(\vec{r})$. Taking a 2D Fourier transform turns this convolution into a point-wise product $\tilde{I}_m(\vec{k})=\tilde{P}_m(\vec{k})\cdot\tilde{O}(\vec{k})$, where every spatial frequency $\vec{k}$ is multiplied by its corresponding transfer coefficient $\tilde{P}_m(\vec{k})$. CLASS/I-CLASS exploit this "diagonal" (i.e., per-$\vec{k}$) structure to retrieve $\tilde{O}$.
        
        In our coherent experiment (Fig. 5), the field at the scattering layer is already naturally represented in k-space:
        $E^{diff}_m(\vec{k})=\tilde{P}_m(\vec{k})\cdot\tilde{O}_{eff}(\vec{k})$,
        where $\tilde{O}_{eff}(\vec{k}) = \tilde{O}(\vec{k})e^{i\phi_{prop}(\vec{k})}$ includes the Fresnel propagation term.
        
        We chose to perform the reconstruction in the diffuser plane since correcting aberrations at their origin provides optimal results. This principle of correcting aberrations in the plane conjugate to where they originate is well-established in adaptive optics (see Choi et al., Nat. Commun. 13, 7939, 2022) and offers optimal correction efficiency.
        
        Since we recover the complete complex wavefront, we can numerically propagate it to any desired axial plane using the Fresnel operator, as detailed in our new Methods section "Fresnel Propagation via Fourier-Domain Transfer Function" and demonstrated in our new supplementary section "Digital Autofocus through Fresnel Propagation."
    \end{ourresponse}



\end{enumerate}

\textbf{Typos:}
\begin{enumerate}[label=\arabic*.]
    \setcounter{enumi}{8}
    \item  \leavevmode\vspace{-\baselineskip}
    \begin{solved_reviewercomment}
        Page 2: “Thus, enabling a straightforward fully interpretable extension of matrix-based methods to rapidly dynamic scatterers, reconstructing complex megapixel-scale images through rapidly varying scattering. Importantly…” This is not a sentence, please rephrase.
    \end{solved_reviewercomment}

    \begin{ourresponse}
        We appreciate the reviewer’s comment and have revised it in the updated manuscript. The revised version now reads:
        
        \begin{quote}
            Thus, our approach provides a natural and fully interpretable extension of matrix-based imaging techniques to the case of rapidly dynamic scatterers. It enables the reconstruction of complex, megapixel-scale images through strongly time-varying scattering. Importantly, unlike state-of-the-art neural-networks-based techniques, our approach does not require assuming a slowly varying medium, making it suitable for dynamic scattering.
        \end{quote}
        
    \end{ourresponse}



    \item  \leavevmode\vspace{-\baselineskip}
    \begin{solved_reviewercomment}
        Page 7: “These experiments’ experimental setup…” Please rephrase.
    \end{solved_reviewercomment}
    \begin{ourresponse}
        We have revised the sentence to read: "The experimental setup and results for these experiments..." which removes the redundancy and improves clarity.
    \end{ourresponse}
    
\end{enumerate}


\subsection*{Reviewer \#2 Comments (from email):}

In this manuscript the authors propose to use the I-CLASS algorithm (which was developed in the same group) to deconvolve the unknown object from the (many) unknown point spread function measured through a dynamic scattering medium. This builds on a previous paper from the same group (ref 14, as far as I can see still unpublished).

\textbf{Does the approach works?} Yes, the results in Fig. 2 alone are convincing enough.

\textbf{Are the result novel?} Yes. The algorithm itself is not new, but its application is.

\textbf{Is the paper written in a way such that somebody knowledgeable in optics but not an ultra-specialist in imaging through scattering media will be able to replicate the results and build up on them?} Not really. 
\begin{enumerate}[label=\arabic*.]

    \begin{solved_reviewercomment}
        The main culprit is that the whole CRT-CLASS/I-CLASS are not well-known algorithms, and the only reference given to readers to make themselves familiar with them is not self-contained. \textbf{This has a very easy fix:} add a section in the supplementary information with an explanation of how and why the algorithm works (there is no space restriction there, so there is also no excuse to provide the necessary information).
    \end{solved_reviewercomment}
    \begin{ourresponse}
        It is the same comment as the first referee.
    \end{ourresponse}

    \item \leavevmode\vspace{-\baselineskip}
    \begin{solved_reviewercomment}
        Another point that is likely to make the life on a non-specialist unnecessarily hard is that eq. 1 is only valid within the isoplanatic patch. For objects larger than the isoplanatic patch nothing of what is presented here will work. This is only briefly addressed at the end of the Discussion section (largely swiping it under the carpet) but if not tackled at the beginning is likely to confuse people.
        
    \end{solved_reviewercomment}
    \begin{ourresponse}
        We thank the reviewer for this suggestion. 
        We have added a clear statement immediately after Equation 1 in the "Principle" section that reads:
        \begin{quote}
            "It is important to note that this convolution model is strictly valid only for objects within an isoplanatic patch. All experiments in this work were designed within this constraint, with potential extensions to larger fields of view or thick complex media, which are discussed in the final section."
        \end{quote}
        
        This addition ensures that readers are aware of this limitation from the outset, helping non-specialists better understand the applicability of our approach without having to wait until the Discussion section.
    \end{ourresponse}
    
    \item A few more minor points:
    \begin{enumerate}[label=3.\alph*.]
        \item \leavevmode\vspace{-\baselineskip}
        \begin{reviewercomment}
            In the introduction the authors claim that iterative phase retrieval lacks guaranteed convergence, which is technically not true. Convergence might take a VERY long time, but it will eventually happen (see \url{https://doi.org/10.1364/AO.21.002758}).
        \end{reviewercomment}
           \hlred{Here we need to answer that it is correct that if you have the correct PSD of the object given some constraint there is a global minimum and you probably eventually converge there, however given non ideal estimation of the PSD (statistical speckle statistics error + noise) it works much worse, regrading statistical noise I would discuss the small PSF regime, and about the shot noise, I would cite Jerome's Optica paper about covariance noise, and claim that since we use the entire matrix and not just the diagonal we have better SNR}
           \\
           \hlred{I recommend applying phase retrieval for millions of iterations with large M to show conergence graph}
        \begin{ourresponse}
        \hlyellow{
            \textbf{Note from Ori:}
            correct phrasing not that convergence is not gurranetedd but that it can take a very long time}
        \end{ourresponse}
        
        \item \leavevmode\vspace{-\baselineskip}
        \begin{solved_reviewercomment}
            Just below the above statement there are two sentences beginning with "while" which looks like they are the leftover of some copy-paste during editing.
        \end{solved_reviewercomment}
        \begin{ourresponse}
            We thank the reviewer for their careful reading. We have removed the redundant sentence and simplified the text to improve clarity and flow. The revised text now reads:
            \begin{quote}
                "While deterministic bispectrum reconstruction can address the convergence challenge of phase retrieval, it still requires averaging a large number of speckle grains, limiting the reconstruction to simple objects."
            \end{quote}
        \end{ourresponse}
        % \hlred{Write something about that it is not really cheating since in incoherent illumination passive constant illumination such as the sun...}
        \item \leavevmode\vspace{-\baselineskip}
        \begin{solved_reviewercomment}
            The results shown in Fig. 2 and 3, albeit impressive, require the target to be illuminated from behind, which makes this whole approach invasive. I know the author never explicitly claim non-invasiveness, but they also never make this point clear.
        \end{solved_reviewercomment}
        \begin{ourresponse}
            We thank the reviewer for highlighting this important point.
            We fully agree that the initial results presented in Figures 2 and 3 require illumination from behind the sample, which makes the approach invasive. 
            
            However, it is worth noting that this illumination geometry is not fundamentally different from standard practices in incoherent imaging. For instance, passive illumination sources like sunlight routinely provide constant, uniform illumination from behind a sample – a technique widely accepted in microscopy and imaging sciences.
            
            To clarify this explicitly in the revised manuscript, we added the following note:
            \begin{quote}
                "\textbf{Note:} The matrix-based imaging approach demonstrated here highlights the fundamental principle of imaging through dynamic scattering media. Although these initial experiments utilize a transmission geometry with illumination from behind the sample, our subsequent experimental demonstrations in fluorescence microscopy (Fig.4) and coherent holographic imaging (Fig.5) showcase the technique's adaptability across diverse optical configurations and imaging modalities."
            \end{quote}
            
            Thus, although the particular results in Figures 2 and 3 are invasive, the manuscript demonstrates a non-invasive adaptation of our method (as shown in Figures 4 and 5).
        \end{ourresponse}
       
        \item \leavevmode\vspace{-\baselineskip}
        \begin{solved_reviewercomment}
            I might have missed it, but I don't think I have seen any discussion about the time needed for the I-CLASS algorithm to converge and give the claimed results.
        \end{solved_reviewercomment}
        \begin{ourresponse}
            Thank you for the opportunity to clarify the computational performance of our algorithm. To give readers a practical sense of the method's efficiency, we've added a runtime specification in the Experimental Parameters section:
            
            \begin{quote}
                The algorithm run time on a commercially available GPU (Nvidia RTX4090, 24 GB) was approximately $\sim 200ms$ per iteration for 150 camera frames at a resolution of $1400 \times 1400$ pixels and around $\sim 50ms$ per iteration for 150 camera frames at a resolution of $700 \times 700$ pixels.
            \end{quote}
        \end{ourresponse}
            
            

        
        \item \leavevmode\vspace{-\baselineskip}
        \begin{solved_reviewercomment}
            For the results shown in Fig. 5 one needs to know the distance between the object and the scattering medium, as only the field at the scattering layer can be reconstructed, and one needs to know how far to propagate it back.
        \end{solved_reviewercomment}

        \begin{ourresponse}
            We thank the reviewer for this insightful question regarding the determination of the correct propagation distance. While the I-CLASS algorithm initially reconstructs the complex field at the scattering layer plane, a key advantage of our coherent imaging approach is that once we have retrieved the complex wavefront, we can numerically propagate it to any desired axial plane.
            
            This "digital autofocus" capability is a fundamental strength of coherent computational imaging. As we now explain in detail in the Methods section titled "Fresnel Propagation via Fourier-Domain Transfer Function," we use the Fresnel propagation transfer function to computationally propagate the reconstructed field from the diffuser plane to any desired plane, including the object plane.
            
            This numerical propagation capability allows us to determine the correct object plane computationally by finding the propagation distance at which the target features achieve optimal focus—eliminating the need for precise a priori knowledge of the object-diffuser distance during data acquisition. This is analogous to the focusing capability in digital holographic microscopy but applied to imaging through scattering media.
            
            To comprehensively address this point, we have added Supplementary Movie S3, which shows the continuous evolution of the reconstructed field as it is propagated from the diffuser plane through the optimal focus plane and beyond. We have also created a new supplementary section titled "Digital Autofocus through Fresnel Propagation" that demonstrates this capability in detail, which we reproduce below for the reviewer's convenience:
        
            \begin{quote}
                \section*{Digital Autofocus through Fresnel Propagation}
                
                In holographic imaging through scattering media, a key advantage is the ability to numerically propagate the reconstructed complex field to any desired axial plane once the field is retrieved at a single reference plane. This capability, often called digital autofocus, eliminates the necessity of knowing the exact object distance during data acquisition.
                
                In our experiments from Fig.~5 of the main text, the I-CLASS algorithm reconstructs the complex wavefront at the scattering layer plane (z = 0). However, as shown in Fig.~S6, this field can be digitally propagated to any desired plane using the Fresnel propagation operator described in the Methods section of the main text. 
                
                By computationally varying the propagation distance and observing the resulting reconstructed intensity distributions, we can identify the optimal object plane where the finest features of the target come into focus. This process is analogous to the physical process of adjusting the focus in a conventional microscope, but performed entirely in post-processing.
                
                Figure~S6 demonstrates this capability by showing the reconstructed field intensity at three distinct propagation distances: before the object plane (z = 5.3 cm), at the object plane where optimal focus is achieved (z = 7.15 cm), and after the object plane (z = 9 cm). The sharp focus observed at z = 7.15 cm confirms that this is indeed the correct object plane, with clear resolution of the fine features in the USAF target.
                
                For a more comprehensive demonstration of this digital autofocus capability, we refer readers to Supplementary Movie S3, which shows the continuous evolution of the reconstructed field as the propagation distance is varied from the diffuser plane (z = 0) to beyond the object plane.
        
                \vspace{1em}
                \begin{minipage}{\linewidth}
                    \centering
                    \includegraphics[width=0.9\textwidth]{figures/figure_S6.pdf}
                    \captionof{figure}{\footnotesize\textbf{Digital autofocus capability through numerical propagation of the reconstructed complex field.} The reconstructed field intensity is shown at three different propagation distances: \textbf{a} z = 5.3 cm (before the object plane), \textbf{b} z = 7.15 cm (at the object plane, where optimal focus is achieved), and \textbf{c} z = 9.0 cm (after the object plane). The sharp focus visible in panel \textbf{b} confirms the correct object plane location, demonstrating that precise knowledge of the object-diffuser distance is not required during data acquisition as optimal focus can be determined computationally during post-processing. Scale bars: 1 mm.}
                \end{minipage}
            \end{quote}
        \end{ourresponse}
        
        
        
        \item \leavevmode\vspace{-\baselineskip}
        \begin{solved_reviewercomment}
            I am not sure I understand how focussing the illumination on the scattering medium can reduce the fluctuations of \(E^{\text{ill}}_m\). Wouldn't it actually maximise them?    
        \end{solved_reviewercomment}
        \begin{ourresponse}
        We thank the reviewer for this insightful question about an important aspect of our experimental design.
        
        The reviewer's intuition that focusing on a scattering medium might maximize illumination fluctuations is valid for many scenarios. However, in our specific configuration, the opposite effect occurs due to a crucial relationship between the illumination spot size and the correlation length of the diffuser.
        
        To address this question comprehensively, we have added a new section ("Effect of illumination spot size on coherent imaging through dynamic scattering") to the supplementary material with numerical simulations demonstrating this effect. For completeness, we reproduce the full section below:
        \begin{quote}
            \section*{Effect of illumination spot size on coherent imaging through dynamic scattering}
        
        In our coherent imaging configuration (Fig. 5 of the main text), maintaining a constant illumination pattern at the object plane despite the dynamic scattering introduced by the rotating diffuser is crucial for successful application of our reconstruction algorithm. As expressed in Eq. 6-7 of the main text, our method requires that the illumination field $E_m^{ill}(\vec{r})$ remains effectively constant across different acquisitions for our method to work properly.
        
        This requirement might appear contradictory, as one might expect that any illumination passing through a dynamic scatterer would inevitably result in varying illumination patterns. However, the key insight is that the relationship between the illumination spot size on the diffuser and the diffuser's correlation length determines whether the illumination at the object plane remains approximately constant between different diffuser realizations.
        
        \begin{minipage}{\linewidth}
        \centering
        \includegraphics[width=0.85\textwidth]{figures/figure_S7.pdf}
        \captionof{figure}{\footnotesize \textbf {Effect of illumination spot size on imaging through dynamic scattering.} Numerical simulations demonstrating how the ratio of illumination spot size to diffuser correlation length affects illumination at the object plane and reconstruction quality. Each column represents a different spot size ratio: 0.1 (left), 1 (middle), and 10 (right). \textbf{a-c} Illumination phase pattern on the diffuser surface. The colorbar indicates phase values from $-\pi$ to $\pi$. The white inset in (a) shows the zoomed-in beam spot. \textbf{d-f} Resulting intensity patterns at the object plane after numerical propagation through the diffuser. The grayscale colorbar indicates normalized intensity. \textbf{g-i} Effective object (target object multiplied by the illumination pattern). \textbf{j-l} Reconstructed images after applying the I-CLASS algorithm. Note how a small spot size relative to the diffuser correlation length (left column) maintains relatively uniform illumination at the object plane, enabling high-quality reconstruction, while larger spot sizes (middle and right columns) create varying speckle patterns that degrade reconstruction quality.}
        \end{minipage}
        
        Fig. S7 demonstrates this principle through numerical simulation. The top row (Fig. S7a-c) shows the phase pattern of the diffuser with the illumination spot superimposed for three cases: where the spot size is 0.1, 1, and 10 times the diffuser correlation length ($d_{\text{correlation}}$), respectively. The second row (Fig. S7d-f) shows the resulting illumination intensity pattern at the object plane after the light has propagated through the diffuser, simulated using Fourier transform propagation.
        
        When the illumination spot is much smaller than the diffuser correlation length (Fig. S7a), it effectively samples only a single "phase patch" of the diffuser. As the diffuser rotates between realizations, this results in a relatively uniform illumination at the object plane that experiences only a global phase shift but maintains its spatial intensity pattern across different realizations. This consistency across different diffuser positions is crucial for our method to work properly. In other words, while the global phase of the illumination might change, its spatial distribution pattern remains effectively identical from one diffuser position to the next, satisfying our requirement for constant spatial illumination.
        
        In contrast, when the illumination spot size is comparable to or larger than the diffuser correlation length (Fig. S7c), the beam simultaneously samples multiple uncorrelated regions of the diffuser. This produces complex speckle patterns at the object plane that vary significantly between diffuser positions, creating a different illumination pattern for each diffuser realization. This variation violates our assumption of constant illumination in Eq. 6-7 of the main text.
        
        The third row (Fig. S7g-i) shows the effect of these illumination patterns on the effective object (the product of the object and the illumination), while the fourth row (Fig. S7j-l) shows the reconstruction results. The reconstruction quality clearly degrades as the spot size increases relative to the diffuser correlation length, demonstrating the critical importance of maintaining consistent illumination across different diffuser realizations.
        
        In our experimental implementation described in the main text, we carefully focused the beam to ensure a spot size smaller than the diffuser's correlation length ($\approx 70 \mu m$), thereby maintaining nearly constant spatial illumination patterns at the object plane across different realizations, with variations limited primarily to global phase shifts.
        \end{quote}
        
        \end{ourresponse}
        
        \item \leavevmode\vspace{-\baselineskip}
        \begin{solved_reviewercomment}
            I am slightly confused by the setup diagram in Fig. 5: The role of the polarizing beam splitter near the scattering medium is clear, as it allows to reject part of the unscattered light, thus maximizing the amount of useful signal, but the polarizing beam splitter closer to the camera seems to ensure that the signal and the reference have opposite polarizations, and thus can never interfere. Am I missing something, or is the diagram wrong?
        \end{solved_reviewercomment}
        
        \begin{ourresponse}
            We thank the reviewer for their careful examination of our experimental setup. The reviewer is correct - this was an error in our diagram. It should indeed be a non-polarizing beam splitter (BS), not a polarizing beam splitter (PBS). We have corrected this in the revised manuscript.
                    
            \vspace{1em}
            \begin{minipage}{\linewidth}
                \centering
                \includegraphics[width=0.99\linewidth]{figures/figure_5.pdf}
                \captionof{figure}{\footnotesize\textbf{Experimental coherent reflection-imaging through dynamic scattering.} 
                \textbf{a} Experimental setup: A reflective target is illuminated through a dynamically rotating scattering diffuser. $M=180$ reflected light fields are holographically recorded in an off-axis holography configuration using a reference arm. %The illumination beam is focused to a tight spot on the diffuser surface such that the object illumination remains relatively constant while the diffuser is rotated.
                \textbf{b} Example of the recorded distorted fields after computational propagation to the object plane. \textbf{c} One example of the recorded field intensity after computational propagation to the object plane. 
                \textbf{d} Reconstructed object intensity at the object plane after applying the I-CLASS algorithm to compensate for scattering, revealing the details of the target. 
                \textbf{e} Complex-valued field amplitude PSFs (APSFs), estimated from the captured fields after effective deconvolution of the reconstructed object field. 
                \textbf{f} Reference intensity image of the object without the diffuser present. Scale bars, 1$mm$}
                \label{fig:myfigure}
            \end{minipage}
        \end{ourresponse}
                
        \item \leavevmode\vspace{-\baselineskip}
        \begin{reviewercomment}
            The comparison between I-CLASS and the phase retrieval algorithms in the supplementary information seems weird. The fact that the phase retrieval is shown to never be able to reconstruct any image more complex than a few dots looks too bad to be true, and is at odds with my personal experience.
        \end{reviewercomment}
        \begin{ourresponse}
            \hlyellow{\textbf{Note from Ori:}
            the reason is that this is not standard phase-retrieval but phase-retrieval of speckle correlation imaging experiment: in a standard phase-retrieval the algorithm is run on the real power spectrum on the object (or equivalently the exact fourier-transform autocorrelation of the object) calculated directly from the object, whereas in our case the phase retrieal is run on an ESTIMATE of the object’s autocorrelation from the speckle correlation. In particular, for speckle correlation imaging to work, the speckled PSF autocorrelation has to resemble a highly-peaked delta-like function. This is the case ONLY when the PSF contains an exteremly large number of speckles, since for a speckle field the autocorrelation peak to background ratio is proportional to the number of speckles contained in the PSF. This practically requires the cpatured scatterd light image to contain a number of speckles that is considerably larger than the number of bright resolution cells in the target, which is not the case in our experiemnts and simulations
            }
            
        \end{ourresponse}
        \hl{example 1}
        
        \hlblue{example 2}
        
        \hlgreen{example 3}
        
        \hlred{example 4}
        
        \hlyellow{example 5}
        
    \end{enumerate}
\end{enumerate}


\end{document}


